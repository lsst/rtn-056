\section{Conclusions} \label{sec:conclusions}
In this report, we have presented a list of sensors that exhibit differences in parameters compared to those obtained by SLAC, including a PTC with low turnoff and inadequate classification. We used the default value to determine turnoff via the \textit{DM stack}, which indicates a decrease of at least 2 points to indicate shutdown.

\vspace{3mm}
Our PTC study revealed a bimodal distribution per vendor for gain and $a_{00}$, and a more generalized behavior for read noise and turnoff. The average gain values we obtained for E2V sensors were $1.49 \pm 0.05$ and for ITL were $1.69 \pm 0.05$ $e^{-}/ADU$. The coefficient of the BF effect had an average value of $(-3.0 \pm 0.1) \times 10 ^{-6}$ for E2V and $(-1.7 \pm 0.2)\times 10 ^{-6}$ for ITL, respectively. Additionally, ITL detectors exhibited higher read noise dispersion compared to E2V, with values of $6.7 \pm 1.0 \; e^{-}$ and $5.4 \pm 0.2 \; e^{-}$, respectively. These results are generally congruent with those obtained by SLAC, although the main difference was observed in the Full Well Capacity value, with our work finding a value of $130000 \pm 10000 \; e^-$, while SLAC reported a value of $90000 \; e^-$, likely due to differences in turnoff calculation methods between \textit{eotest} and \textit{DM stack}.


\vspace{3mm}
Further analysis of the gain obtained from a pair of flats and the gain obtained from the PTC initially revealed a relative percentage error for low fluxes (5K and 10K ADU) higher than 5\%. This prompted a thorough investigation, which revealed that the distribution following the Lupton equation for flat images is not of Gaussian type, leading to truncation of the distribution for statistics calculation, resulting in a shift of the mean value and consequently larger values of gain compared to the PTC gain. To address this, we performed the gain calculation without truncation and obtained percentage differences between these two gains of $(1.8 \pm 0.7, 4.1 \pm 0.9)$ for E2V and $(0.85 \pm 0.7, 2.2 \pm 0.9)$ for ITL, with these intervals corresponding to a flux region between 5000 and 10000 ADU. Accordingly, the respective report was updated and the revised calculation was implemented in the main code.


\vspace{3mm}
Finally, we found that linearity correction effectively fixes the observed bump around 50K-60K ADU, while correction for crosstalk does not significantly affect the shape of the PTC or modify the parameters. Therefore, we recommend implementing linearity correction only.