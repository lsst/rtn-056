\section{Conclusions} \label{sec:conclusions}

We present in this report a list of the sensors that present differences in the parameters with those obtained by SLAC, a PTC with low turnoff, and an inadequate classification of the same so that they can be reviewed later. We use the default value to determine the turnoff via \textit{DM stack}: decrease by at least 2 points to indicate shutdown.

\vspace{3mm}

We find in the PTC study that there is a bimodality per vendor for gain and $a_{00}$ and a more generalized behavior for read noise and turnoff. The average value we found for the gain in the E2V sensors of $1.49 \pm 0.05$ and $1.69 \pm 0.05$ $e^{-}/ADU$ for ITL. The coefficient of the BF effect has an average value of $(-3.0 \pm 0.1)\times 10 ^{-6}$ and $(-1.7 \pm 0.2)\times 10 ^{-6}$ for E2V and ITL, respectively. In addition, ITL detectors present a higher read noise dispersion compared to E2V ($6.7 \pm 1.0 \; e^{-}$ and $5.4 \pm 0.2 \; e^{-}$, respectively). The results obtained in this work are generally congruent with those obtained by SLAC. However, the main difference was observed in the Full Well Capacity value: in this work, we found a value of $130000 \pm 10000$ e$^-$, while SLAC a value of $90000$ e$^-$, which is a product of the different ways of calculating the turnoff between \textit{eotest} and \textit{DM stack}.

\vspace{3mm}

The analysis between the gain obtained from a pair of flats and the gain obtained from the PTC initially yielded a relative percentage error for low fluxes (5K and 10K ADU) higher than 5\%. This result led to a thorough investigation of its origin, finding that the distribution following the Lupton equation for flat images is not of Gaussian type, so the truncation of the distribution to calculate the statistics generating a shift of the mean value, and consequently, larger values of gain concerning the PTC gain. Therefore, this calculation was performed without truncation and obtained percentage differences between these two gains of $(1.8 \pm 0.7, 4.1 \pm 0.9)$ for E2V and $(0.85 \pm 0.7, 2.2 \pm 0.9)$ for ITL, where these intervals correspond to a flow region between 5000 and 10000 ADU. Consequently, the respective report was made and finally implemented in the main code. 


\vspace{3mm}

Finally, the linearity correction fixes the observed bump around 50K-60K ADU. Whereas performing a correction for crosstalk does not affect the shape of the PTC, nor does it significantly modify the parameters. Therefore, we recommend correcting for linearity only.  

